\documentclass[11pt]{letter}
\usepackage[margin=0.9in]{geometry}
\usepackage{parskip}
\usepackage{hyperref}
\usepackage{enumitem}
\setlength{\parskip}{0.8em}

\signature{Roland R. Rodriguez, Jr.}
\address{}

\begin{document}

\begin{letter}{Editors\\
Journal of Autonomous Agents and Multi-Agent Systems\\
Special Issue: ``When Foundation Models Meet Multi-Agent Systems''}

\opening{Dear Editors,}

I am pleased to submit ``Emergent Coordination in Multi-Agent Systems via Pressure Fields and Temporal Decay'' for your special issue. This paper demonstrates that foundation model capabilities and MAS coordination mechanisms are \emph{mutually enabling}: FMs solve the action enumeration problem that limited stigmergic approaches to discrete spaces, while MAS pressure gradients provide principled criteria for combining FM outputs---replacing ad-hoc voting with quality-based selection. Section~7.7 explicitly articulates this FM-MAS reciprocity.

\textbf{Connections to Prior MAS Work.}
The paper situates pressure-field coordination within foundational MAS research:
\begin{itemize}[nosep,topsep=2pt]
\item Unlike Horling \& Lesser's organizational paradigms, pressure-field achieves role-free coordination through shared gradients.
\item Unlike GPGP's explicit task structures and commitment protocols, pressure-field requires no inter-agent messages---$O(1)$ coordination overhead.
\item Unlike SharedPlans and Joint Intentions, pressure-field eliminates intention reasoning; the shared artifact \emph{is} the mutual belief.
\item We compare directly against AutoGen-style conversation baselines, demonstrating 4$\times$ higher solve rates.
\item We extend stigmergic principles to FM-based artifact refinement, showing how FMs overcome the action enumeration limitation.
\end{itemize}

\textbf{Empirical Results.}
Across 1350 trials on meeting room scheduling: pressure-field achieves 30$\times$ higher solve rates than hierarchical control (48.5\% vs 1.5\%), 4$\times$ higher than conversation-based approaches (48.5\% vs 11.1\%), all comparisons highly significant ($p < 0.001$, Cohen's $h > 1.0$). On medium and hard problems, only pressure-field achieves non-zero solve rates. These results challenge the assumption that explicit coordination outperforms implicit coordination.

\textbf{Theoretical Results.}
The paper provides convergence guarantees under pressure alignment (Theorem~1), $O(1)$ coordination overhead (Theorem~4), and proves temporal decay is necessary to escape local minima (Theorem~3).

This work has not been published in a peer-reviewed venue. An earlier arXiv preprint exists; this submission adds FM-MAS reciprocity discussion framed for the special issue theme.

\closing{Sincerely,}

\end{letter}
\end{document}
