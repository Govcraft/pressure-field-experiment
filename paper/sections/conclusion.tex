%% ============================================================================
%% CONCLUSION
%% ============================================================================

\section{Conclusion}

We presented pressure-field coordination, a decentralized approach to multi-agent systems that achieves coordination through shared state and local pressure gradients rather than explicit orchestration.

Our theoretical analysis establishes convergence guarantees under pressure alignment conditions, with coordination overhead independent of agent count. Empirically, on meeting room scheduling across 1350 total trials (270 per strategy), we find:

\begin{enumerate}
\item \textbf{Implicit coordination outperforms explicit coordination}. Pressure-field achieves 48.5\% aggregate solve rate---nearly half of all problems solved through local pressure-following alone, with no coordinator, no message passing, and no explicit task delegation. The gap versus hierarchical control ($30\times$) and conversation-based coordination ($4\times$) is both large and highly significant (all $p < 0.001$).

\item \textbf{Pressure-field is the only strategy that scales to harder problems}. On medium and hard problems, pressure-field achieves 43.3\% and 15.6\% solve rates respectively, while all baselines achieve 0\%.

\item \textbf{Temporal decay shows beneficial effects}. Ablation studies observe a 10 percentage point improvement with decay enabled, consistent with theoretical predictions about escaping local minima, though larger samples would be needed to establish statistical significance.
\end{enumerate}

This work demonstrates that implicit coordination can outperform explicit coordination for constraint satisfaction tasks. Pressure-field achieves this with simpler architecture: no coordinator agent, no explicit message passing, just shared state and local pressure gradients.

\Acp{fm}' zero-shot capabilities eliminate the need for domain-specific action representations; pressure-field coordination eliminates the need for complex multi-agent protocols; together they enable simple multi-agent systems.

These results suggest that for domains with measurable quality signals and locally-decomposable structure, implicit coordination through shared state offers not just a simpler but a more effective alternative to explicit hierarchical control.
